%导言区
\documentclass{article}%book,report,letter
\usepackage{url}
%\usepackage{ctex}
\usepackage{amsmath}
\usepackage{graphicx}
\usepackage{makecell} %实现表格内换行
\usepackage{underlin}
%\graphicspath{{QuestionI.png}}
\newcommand\degree{^\circ}
\newcommand{\myfont}{\textit{\textbf{\textsf{Fancy Text}}}}
\title{\kaishu 牛顿插值}
\author{Shirley}
\date{\today}

%正文区
\begin{document}
	Team \#apmcm2112772
	
	
	5.3.2(9)
	\begin{equation}
		R=(r_{ij})_{m\times n}: \; r_{ij}=w_{j}\cdot w_{ij}(i=1,2,\cdots ,m;j=1,2,\cdots ,n)
	\end{equation}
	5.3.2(10)
	\begin{equation}
		S_{j}^{+}=\left\{\begin{array}{l}
			\underset{1\le i\le m}{max\{r_{i j}\}} ,j=1,2, \cdots, n;  D_{j} \text { is the benefit indicator } \\
			\underset{1\le i\le m}{min\{r_{i j}\}}, j=1,2, \cdots, n;  D_{j} \text { is the cost indicator }
		\end{array}\right.
	\end{equation}
	5.3.2(11)
	\begin{equation}
		S_{j}^{-}=\left\{\begin{array}{l}
			\underset{1\le i\le m}{min\{r_{i j}\}} ,j=1,2, \cdots, n;  D_{j} \text { is the benefit indicator } \\
			\underset{1\le i\le m}{max\{r_{i j}\}}, j=1,2, \cdots, n;  D_{j} \text { is the cost indicator }
		\end{array}\right.
	\end{equation}
	5.3.2(12)
	\begin{equation}
		\left\{\begin{matrix} 
			Sd_{i}^{+}=\sqrt{\sum_{j=1}^{n}(S_{j}^{+}-r_{ij})^{2}},i=1,2,\cdots ,m \\  
			Sd_{i}^{-}=\sqrt{\sum_{j=1}^{n}(S_{j}^{-}-r_{ij})^{2}},i=1,2,\cdots ,m 
		\end{matrix}\right. 
	\end{equation}
	5.3.2(13)
	\begin{equation}
		\eta _{i}=\frac{Sd_{i}^{-}}{Sd_{i}^{+}+Sd_{i}^{-}} ,i=1,2,\cdots ,m
	\end{equation}


\end{document}